\begin{resumo}

\par Podemos encontrar, em várias aplicações práticas incluindo o arrefecimento, problemas de transferência de calor. É necessário uma caracterização precisa dos fenómenos observados para se melhorar a transferência de calor em interfaces, que requer técnicas de diagnostico com elevada resolução temporal e espacial.

\par A Termografia tem mostrado um grande potencial, mas necessita de uma calibração decente. Deve-se também dar especial atenção aos procedimentos de pós-processamento.

\par O trabalho aqui apresentado explora a combinação entre termografia de infravermelhos com elevada resolução temporal e imagiologia de alta velocidade para descrever os fenómenos de transferência de calor que ocorrem na interface durante impactos de gotas em superfícies aquecidas, analisando o efeito de diferentes parâmetros como a molhabilidade, velocidade de impacto e as propriedades dos líquidos (tensão superficial). Para satisfazer a necessidade de uma boa resolução, uma técnica de calibração foi desenhada para melhorar a precisão da câmara. Várias técnicas de pós-processamento foram também usadas para extrair resultados e melhorar a sua qualidade. 

\par A Termografia de Infravermelhos permitiu identificar e descrever em detalhe acontecimentos particulares que foram mencionados na literatura. A calibração proposta melhorou a complacência dos resultados extraído com os fenómenos de transferência de calor. Os resultados mostram que elevada velocidade de impacto, boa molhabilidade e baixa tensão superficial aumentam o fluxo de calor entre a superfície e a gota. Estas conclusões foram também relacionadas com a área molhada pela gota. O fluxo de calor e a eficácia de arrefecimento calculados mostram resultados que estão de acordo com o esperado na literatura.

\end{resumo}