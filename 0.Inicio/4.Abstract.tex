\begin{abstract}
\par Interfacial heat transfer problems are present in various practical situations, including cooling applications. The accurate description of the observed phenomena, required to improve interfacial heat transfer requires accurate diagnostic techniques with high temporal and spatial resolution.

\par Time resolved Thermography has shown high potential to be used but requires a proper camera calibration. Care must be also taken in the post-processing procedure.

\par The present work explores time resolved infrared thermography, combined with high-speed imaging to describe the interface heat phenomena occurring during droplet impacts on heated surfaces, addressing the effect of different parameters such as wettability, impact velocity and liquid properties (surface tension). To comply with the demanded resolution a calibration was designed to improve the camera's precision. Several data processing techniques were also applied to extract the results and improve their quality.

\par Infrared thermography has allowed identifying and describing in detail particular phenomena reported in the literature. The proposed calibration improved the complaisance of the results with the expected heat transfer events. The results show that higher impact velocity, good wettability and low surface tension increase the heat flux between the surface and the impacting droplet. This was also related with the wetted area. The heat flux and cooling effectiveness calculated showed satisfactory values in accordance to the results previously reported in the literature.

\end{abstract}