\section{Objectives}
\label{sec:int_contributions}

\par The main objective of this work is to explore the potential of the use of an high speed infrared camera to study in detail the heat transfer processes occurring at liquid-solid interfaces.\\

\par Here, the proper calibration and post-processing procedures must be developed and tested on a case study. Which was chosen to be the impact of droplets onto heated thin foil surfaces.\\

\par The IR images are analyzed together with high speed images, to relate droplet dynamics with heat transfer processes.\\

\par The calibration and post-processing procedures developed are evaluated based on a critical analysis to infer on how good the data collected by the IR camera can be used to describe in detail the phenomena reported in the literature and provide complementary information on the temperature distributions, heat flux and cooling effectiveness
Finally, the validated data is also used to explore the physics governing the observed phenomena and discuss the effect of droplet impact velocity, wettability, liquid surface tension and temperature.\\

\par The heat transfer processes occur at single phase and when the liquid droplet is boiling.
