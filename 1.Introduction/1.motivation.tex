\section{Motivation}
\label{sec:int_motivation}
% bastante alterado

\par Heat transfer in fluid-solid interfaces, with fluid phase change is a common phenomenon observed  in the nature and relevant for a wide number of industrial applications, namely in cooling systems, based for instance in droplet/spray impact, or pool boiling. The heat transfer mechanisms occurring in such applications are complex and there are still several processes which remain unexplained, despite the numerous studies that have been reported in the literature.\\

\par The accurate description of such phenomena requires the use of diagnostic techniques with high spatial and temporal resolution has they occur in characteristic spatial scales which can be of the order of the micrometers and temporal scales of the order of milliseconds. Within this scope, several diagnostic techniques have been explored although many of them are intrusive and do not comply with the required spatial and temporal resolutions. For instance, the use of a thermocouples is a really common method, but is  intrusive to the measured process, can only measure the surface temperature at one specific location (one point) and cannot be in contact with electrically conductive means. With this in mind, infra-red (IR) thermography has been recently explored as a high potential alternative to some of the existing intrusive temperature measuring methods. A thermographic camera with a proper calibration can give high precision temperature results at high frame rates, which can provide high definition qualitatively and, more importantly, quantitatively accurate thermal images. The IR camera also outputs two dimensional images, a great advantage when trying to understand this kinds of processes, which are usually not restricted to a one dimensional analysis. \\ 

\par However, care must be taken when developing the calibration process and post-processing procedures, as there are many issues that must be considered, related to the dependence of the read temperature values with many parameters (e.g. ambient temperature, effect of the surroundings, surface emissivity, among others). Hence, custom made calibration and post-processing procedures must be developed and explored, to infer, based on a critical analysis, on the accuracy of the provided information and how useful it is to determine additional important features such as accurate temperature distributions, heat fluxes or cooling effectiveness.\\

\par In this context, the present work explores the use of time resolved infrared thermography to describe the heat transfer processes occurring at droplet impacts onto thin metal foils. Although the IR camera use will be centered in the boiling process, the heat transfer mechanisms in droplet surface impact will also be studied.\\

\par While this work was being developed, a complementary computational study is being performed by  Emanuele Teodori, so the results produced in the present work will also be used to validate the computational model that is being devised.